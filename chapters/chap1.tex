% ======================
% FILE: chapters/chap1.tex
% ======================
\chapter{The Market}\label{chap:market}

\section{Introduction}

In microeconomics, we study the behavior of individual economic agents---primarily consumers and firms---and how they interact in markets. We typically build simplified models to understand complex economic phenomena. A key principle in our modeling is the \textbf{optimization principle}, which states that people try to choose the best patterns of consumption that they can afford. Another is the \textbf{equilibrium principle}, where prices adjust until the amount that people demand of something is equal to the amount that is supplied.

In this introductory chapter, we will look at a fundamental concept for evaluating economic outcomes: efficiency. This will provide a criterion to judge how well an economic system performs.

\section{Pareto Efficiency}\label{sec:pareto}

When we want to evaluate the desirability of different economic allocations, we need a standard. One of the most widely used concepts is named after the Italian economist Vilfredo Pareto (1848--1923).

\begin{definition}[Pareto Improvement and Efficiency]
\begin{itemize}
    \item A \textbf{Pareto improvement} is a change to a different allocation that makes at least one individual better off without making any other individual worse off.
    \item If an allocation allows for a Pareto improvement, it is called \textbf{Pareto inefficient}.
    \item If an allocation is such that no Pareto improvements are possible, it is called \textbf{Pareto efficient}.
\end{itemize}
\end{definition}

A Pareto efficient outcome can be thought of as one with ``no wasted welfare''. That is, in a Pareto efficient allocation, the only way to improve one person's welfare is to lower another person's welfare.

\begin{remark}
    \begin{itemize}
    \item A Pareto inefficient outcome implies that there are still unrealized mutual gains-to-trade. It is possible to rearrange the allocation of goods to make someone happier without hurting anyone else.
    \item Any market outcome that achieves all possible gains-to-trade must be Pareto efficient. The concept of Pareto efficiency is a cornerstone of welfare economics.
\end{itemize}
\end{remark}