\chapter{Exchange}

In the previous parts of the course, we have analyzed consumer theory and production theory separately. We are now entering Part 3: \textbf{General Equilibrium Theory}\index{General Equilibrium}. Unlike partial equilibrium analysis, which looks at a single market in isolation, general equilibrium analysis studies how demand and supply conditions interact in several markets to determine the prices of many goods.

We begin our study with the simplest form of general equilibrium: a pure \textbf{exchange economy}\index{Exchange Economy}. In this model, there is no production; economic agents trade their initial endowments of goods to achieve a more preferred allocation.

\section{The Edgeworth Box}

To analyze the exchange of goods between two consumers, Edgeworth and Bowley devised a diagrammatic tool known as the \textbf{Edgeworth Box}\index{Edgeworth Box}. This diagram allows us to depict the endowments, preferences, and possible allocations of two consumers simultaneously.

\subsection{Constructing the Box}
Consider an economy with two consumers, denoted by $A$ and $B$, and two goods, denoted by $1$ and $2$.
\begin{itemize}
    \item Let $\omega^A = (\omega_1^A, \omega_2^A)$ be consumer A's initial endowment.
    \item Let $\omega^B = (\omega_1^B, \omega_2^B)$ be consumer B's initial endowment.
\end{itemize}

The total quantity available of good 1 and good 2 in the economy is fixed by the sum of individual endowments:
\begin{align}
    \text{Total Good 1: } & \quad W_1 = \omega_1^A + \omega_1^B \\
    \text{Total Good 2: } & \quad W_2 = \omega_2^A + \omega_2^B
\end{align}

The Edgeworth box is a rectangle formed by these total quantities:
\begin{itemize}
    \item The \textbf{width} of the box represents the total amount of good 1 ($W_1$).
    \item The \textbf{height} of the box represents the total amount of good 2 ($W_2$).
\end{itemize}

\begin{remark}
    The origin for consumer A ($O_A$) is located at the bottom-left corner. Consumer A's consumption increases as we move to the right (for good 1) and upward (for good 2).
    
    The origin for consumer B ($O_B$) is located at the \textbf{top-right} corner. Crucially, consumer B's axes are reversed: his consumption of good 1 increases as we move to the \textit{left}, and his consumption of good 2 increases as we move \textit{down}.
\end{remark}

\subsection{Feasible Allocations}
An allocation is a specific distribution of goods between the consumers, denoted by $(x_1^A, x_2^A)$ for consumer A and $(x_1^B, x_2^B)$ for consumer B.

\begin{definition}[Feasible Allocation]\index{Feasible Allocation}
    An allocation is \textbf{feasible} if the total amount consumed equals the total amount available:
    \begin{align*}
        x_1^A + x_1^B &= \omega_1^A + \omega_1^B = W_1 \\
        x_2^A + x_2^B &= \omega_2^A + \omega_2^B = W_2
    \end{align*}
\end{definition}

Every point \textit{inside} or on the boundary of the Edgeworth box represents a feasible allocation. The initial \textbf{endowment allocation}\index{Endowment Allocation} is simply one specific point in this box, often marked as $\omega$.

\section{Preferences and Trade}

We can represent the preferences of both consumers by drawing their indifference curves inside the box.
\begin{itemize}
    \item \textbf{Consumer A's Indifference Curves:} Look like standard convex curves relative to $O_A$. Utility increases as curves move further away from $O_A$ (up and to the right).
    \item \textbf{Consumer B's Indifference Curves:} Are convex relative to $O_B$. Utility increases as curves move further away from $O_B$ (down and to the left).
\end{itemize}

\subsection{Pareto Improvement}
Starting from the endowment point $\omega$, consumers can trade. If consumer A gives up some good 1 in exchange for some good 2 from consumer B, they move to a new point in the box.

\begin{definition}[Pareto Improvement]\index{Pareto Improvement}
    An allocation is a \textbf{Pareto improvement} over the initial endowment if:
    \begin{enumerate}
        \item At least one consumer is strictly better off (higher utility), and
        \item The other consumer is strictly no worse off (utility is at least constant).
    \end{enumerate}
\end{definition}

Geometrically, the set of Pareto-improving allocations forms a "lens" shape bounded by the indifference curves of A and B passing through the initial endowment point. Trade will naturally occur within this lens because rational consumers will refuse any trade that lowers their utility.

\section{Pareto Efficiency}

As consumers continue to trade, they will eventually reach a point where no further Pareto improvements are possible.

\begin{definition}[Pareto Efficiency]\index{Pareto Efficiency}
    An allocation is \textbf{Pareto Efficient} (or Pareto Optimal) if there is no way to make one consumer better off without making the other consumer strictly worse off.
\end{definition}

\subsection{The Tangency Condition}
In the interior of the Edgeworth box, Pareto efficiency typically occurs where the indifference curves of the two consumers are \textbf{tangent} to each other.
\begin{itemize}
    \item If the curves intersect (cross each other), there is a "lens" area between them, indicating room for Pareto improvement.
    \item If they are tangent, their slopes are equal.
\end{itemize}

Since the slope of an indifference curve is the Marginal Rate of Substitution (MRS), the condition for Pareto efficiency is:
\begin{equation}
    MRS^A_{1,2} = MRS^B_{1,2}
\end{equation}
Intuitively, this means both consumers value the trade-off between good 1 and good 2 identically at the margin.

\section{The Contract Curve and The Core}

\subsection{The Contract Curve}
There are infinitely many Pareto efficient points in the box. If we connect all such points (all the points of tangency between A's and B's indifference curves), we trace out a line.

\begin{definition}[Contract Curve]\index{Contract Curve}
    The \textbf{Contract Curve} is the set of all Pareto-efficient allocations in the Edgeworth box.
\end{definition}

Note that the contract curve extends from consumer A's origin ($O_A$) to consumer B's origin ($O_B$). Being on the contract curve implies efficiency, but it does not say anything about equity or fairness.

\subsection{The Core}
While the contract curve represents all \textit{possible} efficient outcomes, rational consumers will only agree to trades that leave them at least as well off as they were with their initial endowments. They can always "block" a trade by simply refusing to participate and consuming their endowment.

\begin{definition}[The Core]\index{The Core}
    The \textbf{Core} of an exchange economy is the subset of the contract curve where allocations are Pareto-efficient \textit{and} constitute a welfare improvement (or at least no loss) for both consumers relative to their initial endowments.
\end{definition}

Geometrically, the Core is the segment of the Contract Curve that lies \textit{inside} the lens formed by the indifference curves passing through the endowment point $\omega$. We expect the outcome of rational bargaining or market trade to settle somewhere within the Core.
\chapter{Market Equilibrium and Welfare}

Previously, we determined that rational trade should lead to an allocation in the \textit{Core}. However, the Edgeworth box alone does not tell us exactly \textit{which} point in the Core will be reached. This depends on the trading mechanism. We now analyze the specific mechanism of \textbf{perfectly competitive markets}\index{Competitive Markets}.

\section{Trade in Competitive Markets}

Suppose there is a market mechanism where an "auctioneer" calls out prices $p_1$ and $p_2$ for good 1 and good 2. Consumers $A$ and $B$ are \textbf{price-takers}.

\subsection{The Budget Constraint}
Given prices $(p_1, p_2)$ and the initial endowment $\omega^i = (\omega_1^i, \omega_2^i)$, the value of a consumer $i$'s endowment determines their budget. The budget constraint is:
\begin{equation}
    p_1 x_1^i + p_2 x_2^i = p_1 \omega_1^i + p_2 \omega_2^i
\end{equation}
This line passes through the endowment point $\omega$. The slope of the budget line is $-\frac{p_1}{p_2}$.

\subsection{Optimal Choice}
Each consumer chooses the bundle $(x_1^{*i}, x_2^{*i})$ that maximizes their utility subject to this budget constraint. Geometrically, this occurs where the budget line is tangent to the highest attainable indifference curve.

We define the \textbf{Net Demand} (or Excess Demand) for consumer $i$ and good $j$ as:
\begin{equation}
    e_j^i = x_j^{*i} - \omega_j^i
\end{equation}
\begin{itemize}
    \item If $e_j^i > 0$, the consumer is a net buyer of good $j$.
    \item If $e_j^i < 0$, the consumer is a net seller of good $j$.
\end{itemize}

\section{General Equilibrium}

A general equilibrium requires that supply equals demand in all markets simultaneously.

\begin{definition}[Walrasian Equilibrium]\index{Walrasian Equilibrium}
    A price vector $(p_1^*, p_2^*)$ is a \textbf{General Equilibrium} if it causes both markets to clear:
    \begin{align}
        x_1^{*A} + x_1^{*B} &= \omega_1^A + \omega_1^B \quad (\text{Market for Good 1 clears}) \\
        x_2^{*A} + x_2^{*B} &= \omega_2^A + \omega_2^B \quad (\text{Market for Good 2 clears})
    \end{align}
\end{definition}

If markets do not clear (disequilibrium), prices will adjust. For instance, if there is excess demand for good 2, $p_2$ will rise relative to $p_1$, flattening the budget line until supply matches demand.

\section{Walras' Law}

Walras' Law is a fundamental identity in general equilibrium theory. It states that the value of aggregate excess demand is identically zero for \textit{any} set of prices, not just equilibrium prices.

\subsection{Derivation}
Consider the budget constraint for each consumer. Since preferences are well-behaved, consumers spend their entire income:
\begin{align}
    \text{Consumer A: } & p_1 x_1^{*A} + p_2 x_2^{*A} = p_1 \omega_1^A + p_2 \omega_2^A \\
    \text{Consumer B: } & p_1 x_1^{*B} + p_2 x_2^{*B} = p_1 \omega_1^B + p_2 \omega_2^B
\end{align}

Summing these equations gives the aggregate value of expenditure equals the aggregate value of endowments:
\begin{equation}
    p_1 (x_1^{*A} + x_1^{*B}) + p_2 (x_2^{*A} + x_2^{*B}) = p_1 (\omega_1^A + \omega_1^B) + p_2 (\omega_2^A + \omega_2^B)
\end{equation}

Rearranging terms to group by goods:
\begin{equation}
    p_1 \underbrace{(x_1^{*A} + x_1^{*B} - \omega_1^A - \omega_1^B)}_{z_1(p_1, p_2)} + p_2 \underbrace{(x_2^{*A} + x_2^{*B} - \omega_2^A - \omega_2^B)}_{z_2(p_1, p_2)} = 0
\end{equation}

Here, $z_1$ and $z_2$ represent the aggregate \textbf{excess demand functions} for good 1 and good 2. Thus, we arrive at Walras' Law:

\begin{proposition}[Walras' Law]\index{Walras' Law}
    For any positive prices $(p_1, p_2)$:
    \begin{equation}
        p_1 z_1(p_1, p_2) + p_2 z_2(p_1, p_2) \equiv 0
    \end{equation}
\end{proposition}

\subsection{Implications}
Walras' Law has powerful implications for finding equilibrium:
\begin{enumerate}
    \item \textbf{Market Redundancy:} If the market for good 1 is in equilibrium ($z_1 = 0$), then:
    \[ p_1(0) + p_2 z_2 = 0 \implies z_2 = 0 \]
    Therefore, if $n-1$ markets clear, the $n$-th market must also clear automatically.
    \item \textbf{Inverse Relationship:} An excess supply in one market ($z_1 < 0$) mathematically implies an excess demand in the other market ($z_2 > 0$) to keep the sum zero.
\end{enumerate}

\section{The Fundamental Theorems of Welfare Economics}

We conclude this chapter by linking the market mechanism back to the concept of Pareto Efficiency.

\subsection{The First Fundamental Theorem}
\begin{proposition}[First Welfare Theorem]\index{First Welfare Theorem}
    Given that consumers' preferences are well-behaved, any competitive equilibrium allocation is \textbf{Pareto Efficient}.
\end{proposition}

\textbf{Intuition:} In equilibrium, both consumers maximize utility subject to the \textit{same} prices. Thus:
\[ MRS^A = -\frac{p_1}{p_2} \quad \text{and} \quad MRS^B = -\frac{p_1}{p_2} \]
Since $MRS^A = MRS^B$, the tangency condition for Pareto efficiency is met. This theorem confirms Adam Smith's "Invisible Hand" hypothesis: separate agents pursuing self-interest leads to an efficient social outcome.

\subsection{The Second Fundamental Theorem}
The First Theorem deals with efficiency, but the outcome might be inequitable (e.g., Consumer A gets everything). Can we achieve a \textit{specific}, fair efficient outcome using markets?

\begin{proposition}[Second Welfare Theorem]\index{Second Welfare Theorem}
    Given well-behaved (convex) preferences, any Pareto Optimal allocation can be supported as a competitive equilibrium, provided the initial endowments are redistributed appropriately.
\end{proposition}

\textbf{Implication:} This theorem suggests a separation between \textit{efficiency} and \textit{distribution}. If society prefers a different efficient allocation (e.g., one that is more equal), it does not need to abandon the price mechanism. Instead, the government can transfer wealth (lump-sum transfers) to shift the endowment point $\omega$ to $\omega'$, and then let the market trade to the desired efficient point on the contract curve.

\begin{examplebox}{Summary of Key Concepts}
\begin{itemize}
    \item \textbf{Edgeworth Box:} Tool to analyze exchange.
    \item \textbf{Pareto Efficiency:} Cannot make someone better off without hurting another.
    \item \textbf{Contract Curve:} Locus of all Pareto efficient points.
    \item \textbf{Walras' Law:} Value of excess demands sums to zero.
    \item \textbf{First Welfare Theorem:} Markets $\to$ Efficiency.
    \item \textbf{Second Welfare Theorem:} Transfers + Markets $\to$ Any Efficient Outcome.
\end{itemize}
\end{examplebox}