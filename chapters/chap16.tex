\chapter{Oligopoly}
\label{ch:oligopoly}

\section{Introduction to Industrial Organization}

So far, we have studied two extreme forms of market structure:
\begin{itemize}
    \item \textbf{Perfect Competition:} Many firms, identical products, no market power.
    \item \textbf{Monopoly:} A single firm, complete market power.
\end{itemize}

In this chapter, we turn our attention to \textbf{Oligopoly}, an important case of industrial organization that lies between these two extremes.

\begin{definition}[Market Structures]
    We distinguish industries based on the number of firms:
    \begin{itemize}
        \item A \textbf{monopoly} is an industry consisting of a single firm.
        \item A \textbf{duopoly} is an industry consisting of two firms.
        \item An \textbf{oligopoly} is an industry consisting of a few firms competing with each other.
    \end{itemize}
\end{definition}

The study of oligopoly is particularly interesting because it introduces \textit{strategic interaction}. Unlike a monopolist (who plays against the market demand) or a perfect competitor (who ignores other firms), an oligopolist must consider how its rivals will react to its decisions.

We will explore different models based on the strategic variable (quantity vs. price) and the order of play (simultaneous vs. sequential).

\section{Quantity Competition: The Cournot Model}

The Cournot model describes a static game where firms compete by choosing \textbf{output levels} (quantities) simultaneously.

\subsection{The Setup}
Consider a duopoly. Let $y_1$ and $y_2$ be the output levels of firm 1 and firm 2, respectively.
\begin{itemize}
    \item Firm 1 takes firm 2's output choice, $y_2$, as given (fixed).
    \item Firm 1's profit function is:
    \begin{equation}
        \Pi_1(y_1; y_2) = p(y_1 + y_2)y_1 - c_1(y_1)
    \end{equation}
    where $p(y_1 + y_2)$ is the market inverse demand function dependent on total output $y_T = y_1 + y_2$.
\end{itemize}

\subsection{Reaction Functions}
Firm 1 seeks to maximize profit given its belief about $y_2$. The first-order condition (FOC) for profit maximization is:
\begin{equation}
    \frac{\partial \Pi_1}{\partial y_1} = p(y_1 + y_2) + y_1 \frac{\partial p(y_1 + y_2)}{\partial y_1} - c_1'(y_1) = 0
\end{equation}
This equates marginal revenue to marginal cost. The solution to this maximization problem is firm 1's \textbf{reaction function} (or best response function):
\begin{equation}
    y_1 = R_1(y_2)
\end{equation}
Similarly, firm 2 maximizes $\Pi_2(y_2; y_1) = p(y_1 + y_2)y_2 - c_2(y_2)$. Its FOC yields firm 2's reaction function:
\begin{equation}
    y_2 = R_2(y_1)
\end{equation}

\subsection{Cournot-Nash Equilibrium}
A Cournot-Nash equilibrium is a set of output levels $(y_1^*, y_2^*)$ such that each firm is maximizing profit given the output of the other. Mathematically, the equilibrium must satisfy both reaction functions simultaneously:
\begin{equation}
    y_1^* = R_1(y_2^*) \quad \text{and} \quad y_2^* = R_2(y_1^*)
\end{equation}
Graphically, this is the intersection of the two reaction curves in the $(y_1, y_2)$ plane.

\begin{examplebox}{Example: Cournot Competition}
\textbf{Setup:}
Assume the market inverse demand function is linear:
\[ p(y_T) = 60 - y_T \]
The total cost functions for the two firms are:
\[ c_1(y_1) = y_1^2 \quad \text{and} \quad c_2(y_2) = 15y_2 + y_2^2 \]

\textbf{Step 1: Firm 1's Optimization}
Given $y_2$, firm 1's profit is:
\[ \Pi_1 = (60 - y_1 - y_2)y_1 - y_1^2 = 60y_1 - y_1^2 - y_1y_2 - y_1^2 = 60y_1 - y_1y_2 - 2y_1^2 \]
First-order condition ($\frac{\partial \Pi_1}{\partial y_1} = 0$):
\[ 60 - y_2 - 4y_1 = 0 \implies 4y_1 = 60 - y_2 \]
Firm 1's reaction function:
\[ y_1 = R_1(y_2) = 15 - \frac{1}{4}y_2 \]

\textbf{Step 2: Firm 2's Optimization}
Given $y_1$, firm 2's profit is:
\[ \Pi_2 = (60 - y_1 - y_2)y_2 - (15y_2 + y_2^2) = 60y_2 - y_1y_2 - y_2^2 - 15y_2 - y_2^2 \]
Simplifying: $\Pi_2 = 45y_2 - y_1y_2 - 2y_2^2$.
First-order condition ($\frac{\partial \Pi_2}{\partial y_2} = 0$):
\[ 45 - y_1 - 4y_2 = 0 \implies 4y_2 = 45 - y_1 \]
Firm 2's reaction function:
\[ y_2 = R_2(y_1) = \frac{45 - y_1}{4} \]

\textbf{Step 3: Finding Equilibrium}
Substitute $R_2$ into $R_1$:
\[ y_1 = 15 - \frac{1}{4} \left( \frac{45 - y_1}{4} \right) \]
\[ y_1 = 15 - \frac{45}{16} + \frac{y_1}{16} \]
\[ \frac{15}{16}y_1 = 15 - 2.8125 = 12.1875 \]
Solving for quantities:
\[ y_1^* = 13 \]
Substitute $y_1^*$ back into $R_2$:
\[ y_2^* = \frac{45 - 13}{4} = \frac{32}{4} = 8 \]

\textbf{Result:} The Cournot-Nash equilibrium is $(y_1^*, y_2^*) = (13, 8)$.
\end{examplebox}

\section{Iso-Profit Curves}

To understand the incentives in oligopoly, we can analyze \textbf{iso-profit curves}. An iso-profit curve for firm 1 describes all pairs of $(y_1, y_2)$ that yield the same level of profit for firm 1.

Consider a general linear demand $p(y_1+y_2) = a - b(y_1+y_2)$ and zero costs for simplicity. Firm 2's profit is:
\[ \Pi_2 = (a - b y_1 - b y_2)y_2 = ay_2 - by_1 y_2 - by_2^2 \]
Rearranging for $y_1$ to plot the curve:
\[ y_1 = \frac{a}{b} - y_2 - \frac{\bar{\Pi}_2}{by_2} \]

\begin{remark}[Shape of Iso-Profit Curves]
For firm 1, profit increases as $y_2$ decreases (competitor produces less). Thus, lower iso-profit curves (closer to the $y_1$ axis) represent higher profits for firm 1. The shape is an inverted "U".
\end{remark}

\textbf{Key geometric properties:}
\begin{itemize}
    \item The peak ("top") of any iso-profit curve for firm 1 lies exactly on firm 1's reaction curve.
    \item Why? Because for any fixed $y_2$, firm 1 chooses $y_1$ to maximize profit, which corresponds to the highest attainable point on the vertical section (the tangency of the horizontal line $y_2=c$ and the iso-profit curve).
    \item The Cournot equilibrium is the intersection of the two reaction curves.
\end{itemize}

\section{Collusion and Cartels}

Are the Cournot-Nash equilibrium profits the largest that firms can earn in total? The answer is generally \textbf{no}.

\subsection{The Incentive to Collude}
Firms can mutually increase profits by coordinating to lower total output, moving closer to the monopoly outcome. This cooperation is called \textbf{collusion}, and firms that collude form a \textbf{cartel}.

The goal of a cartel is to choose $(y_1, y_2)$ to maximize joint profit:
\[ \Pi^m(y_1, y_2) = p(y_1+y_2)(y_1+y_2) - c_1(y_1) - c_2(y_2) \]

Graphically, the joint profit maximizing point $(y_1^m, y_2^m)$ is where the iso-profit curves of the two firms are tangent to each other. At the Cournot equilibrium, the iso-profit curves intersect, implying there is a "lens" shaped region between them where both firms can achieve higher profits (Pareto improvement).

\subsection{The Instability of Cartels}
While cartels maximize joint profits, they are fundamentally unstable due to the \textbf{incentive to cheat}.

Suppose firms agree to produce the cartel output $(y_1^m, y_2^m)$.
\begin{itemize}
    \item If firm 1 produces $y_1^m$, firm 2's best response is $R_2(y_1^m)$.
    \item Since $y_2^m$ restricts output to keep prices high, generally $R_2(y_1^m) > y_2^m$.
    \item Firm 2 can increase its \textit{individual} profit by unilaterally increasing output (cheating).
\end{itemize}

Mathematically, the first-order conditions for the cartel (joint maximization) require:
\[ \frac{\partial \Pi^m}{\partial y_1} = p(y_T) + y_T \frac{\partial p}{\partial Y} - MC_1(y_1) = 0 \]
However, if firm 1 maximizes only its own profit $\Pi_1$, the condition is:
\[ \frac{\partial \Pi_1}{\partial y_1} = p(y_T) + y_1 \frac{\partial p}{\partial Y} - MC_1(y_1) = 0 \]
Comparing these, since $\frac{\partial p}{\partial Y} < 0$ and $y_T > y_1$, the marginal revenue perceived by the individual firm is higher than the marginal revenue perceived by the cartel. Therefore:
\[ \frac{\partial \Pi_1}{\partial y_1} \bigg|_{(y_1^m, y_2^m)} > 0 \]
The individual firm always has a marginal incentive to increase output at the cartel solution. This leads to broken agreements (e.g., OPEC).

\section{Stackelberg Games: Sequential Competition}

In the Cournot model, firms move simultaneously. What if one firm chooses its output first (the \textbf{Leader}) and the other responds (the \textbf{Follower})? This is a \textbf{Stackelberg game}.

\subsection{Backward Induction}
To solve this, we use backward induction:
\begin{enumerate}
    \item \textbf{Follower's Problem:} Firm 2 observes firm 1's choice $y_1$ and chooses $y_2$ to maximize its own profit. This is simply firm 2's reaction function:
    \[ y_2 = R_2(y_1) \]
    \item \textbf{Leader's Problem:} Firm 1 anticipates this reaction. It chooses $y_1$ to maximize $\Pi_1(y_1, R_2(y_1))$.
\end{enumerate}

The leader's profit function becomes:
\[ \Pi_1^S(y_1) = p(y_1 + R_2(y_1))y_1 - c_1(y_1) \]

\begin{examplebox}{Example: Stackelberg Competition}
Using the same functions as the Cournot example:
\[ p = 60 - y_T, \quad c_1 = y_1^2, \quad c_2 = 15y_2 + y_2^2 \]

\textbf{1. Follower's Response (Firm 2)}
From the previous section, we know firm 2's reaction function is:
\[ y_2 = \frac{45 - y_1}{4} \]

\textbf{2. Leader's Optimization (Firm 1)}
Firm 1 substitutes $y_2$ into its profit function:
\begin{align*}
    \Pi_1 &= (60 - y_1 - y_2)y_1 - y_1^2 \\
          &= \left( 60 - y_1 - \frac{45 - y_1}{4} \right)y_1 - y_1^2
\end{align*}
Simplify the term in the brackets:
\[ 60 - \frac{45}{4} - y_1 + \frac{y_1}{4} = \frac{240-45}{4} - \frac{3}{4}y_1 = \frac{195}{4} - \frac{3}{4}y_1 \]
Now substitute back into profit:
\[ \Pi_1 = \left( \frac{195}{4} - \frac{3}{4}y_1 \right)y_1 - y_1^2 = \frac{195}{4}y_1 - \frac{3}{4}y_1^2 - \frac{4}{4}y_1^2 = \frac{195}{4}y_1 - \frac{7}{4}y_1^2 \]
First-order condition ($\frac{d \Pi_1}{d y_1} = 0$):
\[ \frac{195}{4} - \frac{14}{4}y_1 = 0 \implies 14y_1 = 195 \]
\[ y_1^S = \frac{195}{14} \approx 13.9 \]
(Note: The slides approximate this as $13.9$).

\textbf{3. Follower's Output}
\[ y_2^S = \frac{45 - 13.9}{4} = \frac{31.1}{4} \approx 7.8 \]
\end{examplebox}

\begin{proposition}[Stackelberg vs. Cournot]
Comparing the results:
\begin{itemize}
    \item Cournot: $y_1^* = 13$, $y_2^* = 8$.
    \item Stackelberg: $y_1^S \approx 13.9$, $y_2^S \approx 7.8$.
\end{itemize}
The leader produces \textbf{more} than its Cournot output, and the follower produces \textbf{less}. The leader earns higher profits than in the Cournot equilibrium because it can manipulate the follower's reaction.
\end{proposition}

\section{Price Competition: The Bertrand Model}

What if firms compete using \textbf{price} strategies instead of quantity strategies? This is known as the Bertrand model.

\subsection{Assumptions and Equilibrium}
\begin{itemize}
    \item Firms produce identical (homogeneous) products.
    \item Firms set prices simultaneously.
    \item Consumers buy from the firm with the lower price.
    \item Constant marginal cost $c$.
\end{itemize}

If Firm 1 sets $p_1 > p_2 > c$, Firm 1 sells nothing.
If Firm 1 sets $p_1 = p_2 > c$, they split the market.
However, if $p > c$, any firm has an incentive to slightly undercut the rival (e.g., set price $p_2 - \epsilon$) to capture the entire market. This "price war" drives the price down to the marginal cost.

\begin{proposition}[Bertrand Equilibrium]
In the unique Nash equilibrium of the Bertrand game with homogeneous goods and constant marginal costs, all firms set their price equal to marginal cost:
\[ p_1 = p_2 = c \]
Economic profits are zero.
\end{proposition}

This result is striking because it suggests that with just two firms, price competition can result in the perfect competition outcome, provided goods are homogeneous and capacity is unlimited.
