\documentclass[12pt, a4paper]{article}
\usepackage{amsmath, amssymb}
\usepackage{geometry}
\geometry{a4paper, total={170mm,257mm}, left=20mm, top=20mm}
\usepackage{palatino} % Using a classic font for the academic feel

\title{\textbf{Intermediate Microeconomics \\ Midterm Examination}}
\author{Mochiao Chen}
\date{Fall Semester}

\begin{document}

\maketitle

\noindent\rule{\linewidth}{0.4pt}
\begin{center}
    \textbf{Instructions}
\end{center}
\begin{itemize}
    \item This exam consists of 6 questions, for a total of 100 points.
    \item Answer all questions. Show all your work and reasoning clearly. Partial credit will be awarded for correct steps.
    \item You may refer to concepts and formulas presented in the course notes.
    \item The points for each question are indicated in brackets.
\end{itemize}
\noindent\rule{\linewidth}{0.4pt}

\vspace{1cm}

\section*{Question 1: Budget Constraints and Preferences [15 points]}

A student has a monthly income of $m = \$600$ to spend on two goods: data for their phone ($x_1$, measured in GB) and all other goods ($x_2$, measured in dollars, so $p_2 = \$1$). The price of data is initially $p_1 = \$10$ per GB.

\begin{enumerate}
    \item[(a)] \textbf{[5 points]} Write down the student's initial budget constraint and draw it on a graph, clearly labeling the intercepts. Now, the government introduces a subsidy program for students. The first 20 GB of data are subsidized at \$5 per GB, but any data consumed beyond 20 GB is at the original market price. Draw the new budget constraint on the same graph, clearly identifying the slope(s) and any kink point.

    \item[(b)] \textbf{[5 points]} State the three fundamental axioms of rational choice (Completeness, Reflexivity, Transitivity). Using the axiom of transitivity, provide a formal proof that two distinct indifference curves for a single individual cannot intersect.

    \item[(c)] \textbf{[5 points]} A consumer's preferences for coffee ($c$) and tea ($t$) can be described by the utility function $U(c, t) = \min\{2c, 3t\}$. What type of preferences does this represent? Draw a representative indifference curve. What is the relationship between the goods?
\end{enumerate}


\section*{Question 2: Utility Maximization and Demand [20 points]}

A consumer's preferences for goods $x_1$ and $x_2$ are represented by the Cobb-Douglas utility function $U(x_1, x_2) = x_1^a x_2^b$, where $a, b > 0$. The consumer has income $m$ and faces prices $p_1$ and $p_2$.

\begin{enumerate}
    \item[(a)] \textbf{[8 points]} Set up the consumer's utility maximization problem. Using the Lagrange method or the tangency condition, derive the Marshallian demand functions for good 1, $x_1^*(p_1, p_2, m)$, and good 2, $x_2^*(p_1, p_2, m)$.

    \item[(b)] \textbf{[6 points]} Based on your derived demand functions:
    \begin{itemize}
        \item Is good 1 a normal good or an inferior good? Explain.
        \item Is good 1 an ordinary good or a Giffen good? Explain.
        \item Are goods 1 and 2 gross substitutes or gross complements? Explain.
    \end{itemize}

    \item[(c)] \textbf{[6 points]} Derive the indirect utility function, $V(p_1, p_2, m)$, by substituting the Marshallian demands back into the utility function. Show that the consumer spends a fixed fraction of their income on each good.
\end{enumerate}


\section*{Question 3: Slutsky Equation and Price Effects [20 points]}

Consider a consumer with preferences for good 1 ($x_1$) and good 2 ($x_2$) given by the quasilinear utility function $U(x_1, x_2) = 2\sqrt{x_1} + x_2$. The consumer's income is $m = \$100$, and prices are initially $p_1 = \$1$ and $p_2 = \$1$.

\begin{enumerate}
    \item[(a)] \textbf{[5 points]} Find the consumer's initial optimal consumption bundle $(x_1^*, x_2^*)$.

    \item[(b)] \textbf{[10 points]} Now, suppose the price of good 1 increases to $p'_1 = \$2$. Calculate the new optimal bundle. Then, decompose the total change in the demand for good 1 ($\Delta x_1$) into the Substitution Effect and the Income Effect using the Slutsky method. You must calculate the numerical values for all three components.

    \item[(c)] \textbf{[5 points]} Illustrate your answer from part (b) on a clear graph. Your graph should show the initial budget line, the new budget line, the compensated (pivoted) budget line, and the optimal bundles A, B, and C corresponding to your decomposition.
\end{enumerate}

\newpage

\section*{Question 4: Revealed Preference [15 points]}

An economist observes a consumer's choices at three different price-income situations:
\begin{itemize}
    \item \textbf{Situation A:} When prices are $p^A = (\$2, \$2)$, the consumer chooses bundle $x^A = (10, 5)$.
    \item \textbf{Situation B:} When prices are $p^B = (\$3, \$1)$, the consumer chooses bundle $x^B = (5, 15)$.
    \item \textbf{Situation C:} When prices are $p^C = (\$1, \$3)$, the consumer chooses bundle $x^C = (15, 5)$.
\end{itemize}

\begin{enumerate}
    \item[(a)] \textbf{[8 points]} Define the Weak Axiom of Revealed Preference (WARP). Systematically check if this consumer's choices are consistent with WARP. Show all necessary calculations and explain any violations you find.

    \item[(b)] \textbf{[7 points]} Define the Strong Axiom of Revealed Preference (SARP). Explain how it differs from WARP. Based on your analysis in part (a), is it possible for these choices to violate SARP? (You do not need to perform new calculations if your answer to (a) is conclusive). Explain why a set of choices that satisfies SARP can be rationalized by a well-behaved utility function.
\end{enumerate}


\section*{Question 5: Endowments and Intertemporal Choice [15 points]}

\begin{enumerate}
    \item[(a)] \textbf{[8 points]} A worker is endowed with $\bar{R}=24$ hours of time per day, which they can allocate between leisure ($R$) and labor ($L = \bar{R} - R$). The wage rate is $w$, and the price of the consumption good ($C$) is $p$. The worker has no non-labor income. Write down the worker's budget constraint in terms of $C$ and $R$. What is the economic interpretation of the slope of this budget constraint? Using the logic of the Slutsky equation with endowments, explain the conditions under which a higher wage rate ($w$) could lead the worker to supply \emph{less} labor (a backward-bending labor supply curve). Be sure to discuss the income and substitution effects.

    \item[(b)] \textbf{[7 points]} A consumer has an income of $m_1 = \$100,000$ in period 1 and expects an income of $m_2 = \$110,000$ in period 2. The nominal interest rate is $r=10\%$. There is an expected inflation rate of $\pi=4\%$.
    \begin{itemize}
        \item What is the present value of the consumer's lifetime income?
        \item What is the future value of the consumer's lifetime income?
        \item What is the real interest rate, $\rho$? Write down the consumer's budget constraint in terms of real consumption, $c_1$ and $c_2$.
    \end{itemize}
\end{enumerate}


\section*{Question 6: Choice Under Uncertainty [15 points]}

An individual has initial wealth of $W = \$100$. There is a 25\% probability ($\pi_b = 0.25$) of a "bad" state occurring, in which they will suffer a loss of $L = \$64$. Their preferences under uncertainty are described by the von Neumann-Morgenstern utility function $U(W) = \sqrt{W}$.

\begin{enumerate}
    \item[(a)] \textbf{[3 points]} What is this individual's expected utility if they do not purchase any insurance? Is this individual risk-averse, risk-neutral, or risk-loving? Justify your answer based on the utility function.

    \item[(b)] \textbf{[7 points]} Suppose the individual can buy insurance at an actuarially fair price. What is the price per dollar of coverage, $\gamma$? What is the optimal amount of insurance coverage, $K^*$, they will buy? Show that they achieve full insurance and calculate their final wealth and utility.

    \item[(c)] \textbf{[5 points]} Now suppose the insurance is sold by a company that charges a price of $\gamma' = \$0.30$ per dollar of coverage. Will the individual still fully insure? Set up the optimality condition and briefly explain the direction of their choice without necessarily solving for the exact new level of coverage.
\end{enumerate}

\section*{Question 7: Consumer's Surplus, Compensating and Equivalent Variation [20 points]}

A consumer's preferences are represented by the utility function $U(x_1, x_2) = x_1 x_2$. Their income is $m = \$120$. Initially, the prices are $p_1 = \$4$ and $p_2 = \$1$. The Marshallian demand functions for this utility are $x_1(p_1, m) = \frac{m}{2p_1}$ and $x_2(p_2, m) = \frac{m}{2p_2}$. The expenditure function is $e(p_1, p_2, u) = 2\sqrt{p_1 p_2 u}$.

\begin{enumerate}
    \item[(a)] \textbf{[4 points]} Calculate the consumer's initial utility level.

    \item[(b)] \textbf{[6 points]} The price of good 1 falls to $p'_1 = \$1$. Calculate the change in consumer's surplus ($\Delta CS$) resulting from this price change. (Hint: The inverse demand function is $p_1(x_1) = \frac{m}{2x_1}$).

    \item[(c)] \textbf{[5 points]} Calculate the Compensating Variation (CV) for this price change. Based on the definition, should this be a positive or negative value for the consumer?

    \item[(d)] \textbf{[5 points]} Calculate the Equivalent Variation (EV) for this price change. For a normal good, what is the expected relationship between the absolute values of $\Delta CS$, CV, and EV? Does your result confirm this relationship?
\end{enumerate}


\section*{Question 8: Corner Solutions and Index Numbers [20 points]}

\begin{enumerate}
    \item[(a)] \textbf{[8 points]} A consumer considers Android phones ($x_1$) and iPhones ($x_2$) to be perfect substitutes. Their utility function is $U(x_1, x_2) = x_1 + 2x_2$. Their budget is $m = \$1200$.
    \begin{itemize}
        \item Derive the consumer's demand function for iPhones, $x_2^*(p_1, p_2, m)$. (This will be a function with different cases depending on the prices).
        \item If the price of an Android phone is $p_1 = \$500$ and the price of an iPhone is $p_2 = \$900$, what bundle will the consumer purchase?
    \end{itemize}

    \item[(b)] \textbf{[12 points]} A consumer's consumption bundles and the corresponding prices for two years are given below:
    \begin{itemize}
        \item \textbf{Year 1 (Base Period):} Prices $p^1 = (\$10, \$20)$. Bundle chosen $x^1 = (20, 10)$.
        \item \textbf{Year 2 (Current Period):} Prices $p^2 = (\$15, \$15)$. Bundle chosen $x^2 = (15, 20)$.
    \end{itemize}
    Calculate the Laspeyres Price Index ($L_p$) and the Paasche Quantity Index ($P_q$). Using the logic of revealed preference, what can you conclude about the change in the consumer's welfare from Year 1 to Year 2? Explain your reasoning for each index.
\end{enumerate}


\section*{Question 9: Asset Markets, Arbitrage, and Resource Depletion [15 points]}

\begin{enumerate}
    \item[(a)] \textbf{[7 points]} Explain the "no-arbitrage principle" in asset markets. Suppose you own a government bond that pays a fixed coupon of \$50 per year and has a current market price of \$1000. The current risk-free interest rate (e.g., from a savings account) is 5\%. Now, the central bank raises the risk-free interest rate to 6\%. What will happen to the market price of your existing bond? Explain the arbitrage process that leads to this outcome.

    \item[(b)] \textbf{[8 points]} The value of a non-depletable resource (e.g., a forest) over time is given by the function $V(t) = 1000 + 400t - 5t^2$, where $t$ is time in years. The market interest rate is constant at $r = 5\%$.
    \begin{itemize}
        \item Write down the condition for the optimal time to sell (harvest) the resource.
        \item Calculate the optimal selling time, $t^*$.
        \item What is the value of the resource at its peak, and at what time does this peak occur? Explain why it is not optimal to hold the resource until its value is maximized.
    \end{itemize}
\end{enumerate}


\section*{Question 10: Theoretical Concepts and Short Proofs [15 points]}

Answer the following three parts.

\begin{enumerate}
    \item[(a)] \textbf{[5 points]} Define Pareto Efficiency. Consider an economy with two individuals, A and B, and two goods, X and Y. The initial allocation is: A has (10 units of X, 2 units of Y) and B has (2 units of X, 10 units of Y). The Marginal Rate of Substitution for A at this point is $MRS_{XY}^A = 3$, and for B it is $MRS_{XY}^B = 1/2$. Is this initial allocation Pareto efficient? If not, propose a trade that would be a Pareto improvement. Explain your reasoning.

    \item[(b)] \textbf{[5 points]} What does it mean for preferences to be homothetic? Prove that if a consumer has homothetic preferences, their income offer curve (income expansion path) is a straight line through the origin.

    \item[(c)] \textbf{[5 points]} Using the Slutsky decomposition, explain the precise conditions under which a Giffen good can exist. Why are Giffen goods so rarely observed in the real world?
\end{enumerate}

\end{document}